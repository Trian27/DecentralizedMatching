\documentclass{article}
\usepackage[utf8]{inputenc}
\usepackage{amssymb, amsmath, amsthm, amsfonts, cancel, hyperref, tikz}
\usepackage[margin=0.75in]{geometry}
\newcommand{\brac}[1]{\left\{ #1\right\}}
\newcommand{\un}[1]{\underline{#1}}
\newtheorem{theorem}{Theorem}
\newtheorem{definition}{Definition}
\begin{document}
\title{Appendix}
\date{}
\author{}
\maketitle
\begin{definition}[Buyers/Sellers]
    Each buyer, $B_i$, has a true cost for the item they are buying, $c_i$, and a bid $b_i$ such 
    that $b_i\ge c_i$. Each seller, $S_j$, has a true price for the item they are selling, $p_j$, and an ask $a_j$ such that $a_j\le p_j$.
    Buyers and sellers also have a UUID unique with respect to their type. For example, if there are 5 buyers and 5 sellers, 
    the buyers will have UUIDs $0,1,2,3,4$ and the sellers will have UUIDs $0,1,2,3,4$.
    In the program we represent the buyer $B_i$ as a list $[u_i,c_i,b_i]$ such that $B_i[0]=u_i$ is the UUID of the buyer, $B_i[1]=c_i$ is the true cost of the buyer, and $B_i[2]=b_i$ is the bid of the buyer.
    We represent the seller $S_j$ as a list $[v_j,p_j,a_j]$ such that $S_j[0]=v_j$ is the UUID of the seller, $S_j[1]=p_j$ is the true price of the seller, and $S_j[2]=a_j$ is the ask of the seller.
\end{definition}
\begin{definition}[Clearing Price]
    Consider some bid $b$ and ask $a$. The clearing price, $cp(b,a)$, a function of the bid and ask that maps to the price at which the transaction clears.
    We require that $cp(b,a)\in [a,b]$.
\end{definition}
\begin{definition}[Surplus]
    Let $B_i$ denote a buyer with cost $c_i$ and bid $b_i$. Let $S_j$ denote a seller with price $p_j$ and ask $a_j$.
    Consider some clearing price $cp(b_i,a_j)$. The surplus for the buyer is defined as $s_{B_i}=c_i-cp(b_i,a_j)$ 
    and the surplus for the seller is defined as $s_{S_j}=p_j-cp(b_i,a_j)$. The total surplus of the transaction, 
    defined $s_{B_i,S_j}=s_{B_i}+s_{S_j}$, is the sum of the surplus for the buyer and the seller. The surplus of any buyer or 
    seller not involved in a transaction is $0$.
\end{definition}
\begin{theorem}[Surplus Calculation]
    The calculation of surplus in the market does not depend on the clearing price.
\end{theorem}
\begin{proof}
    We define the surplus of the market as the summation of the surplus for each transaction.
    Note that for each transaction with buyer and seller $B_i,S_j$, the surplus is defined as $s_{B_i,S_j}=c_i-p_j$
    and thus does not depend on the clearing price.
\end{proof}
\end{document}